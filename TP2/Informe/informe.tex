\documentclass[onecolumn,10pt]{article}
\linespread{1.3}
\usepackage[margin=2cm]{geometry}

\usepackage[utf8]{inputenc}
\usepackage[spanish]{babel}
\usepackage{tikz}
\usepackage{paralist}
\usepackage{subcaption} \usepackage{graphicx}
\usepackage{amsmath} \usepackage{amssymb}
\usepackage{xcolor}
\usepackage{listings}

\usepackage{amsmath}
\usepackage{algorithm}
\usepackage[noend]{algpseudocode}

\lstset{language=C}
\bibliographystyle{apsrev4-1}

\graphicspath{{./fig/}}

\newcommand{\IDT}{\texttt{IDT}}
\newcommand{\MMU}{\texttt{MMU}}
\newcommand{\GDT}{\texttt{GDT}}
\newcommand{\TSS}{\texttt{TSS}}
\newcommand{\PD}{\texttt{PD}}

\begin{document} 

\title{Trabajo Práctico 2: Redes Neuronales Artificiales}

\author{L. Alvarez, P. Bordon y D. Santos}

\date{\today}

%\begin{abstract}
%\end{abstract}

\maketitle


\newpage

\tableofcontents

\newpage

\section{Descripción del problema}
\section{Descripción del problema}

\par Se cuenta con documentos de descripción de empresas. Dichos documentos se encuentran preprocesados, de manera tal que se tiene un dataset que especifica qué cantidad de veces aparece cada palabra en cada documento. Este formato es conocido como Bolsa de Palabras (Bag-Of-Words). El problema a resolver será diseñar y entrenar una red neuronal mediante métodos de aprendizaje hebbiano no supervisado para que pueda clasificar automáticamente los documentos según categoría. También se busca que esta red pueda generalizar corréctamente y responda de manera certera a nuevos documentos.


\par Se utilizaron dos subparadigmas clásicos de aprendizaje no supervisado: aprendizaje hebbiano y aprendizaje competitivo. Se trabajó en un modelo de reducción de dimensiones y otro de mapeo de características auto\-organizado para su aplicación sobre las datos propuestos. En el primer caso se planteó una reducción a tres dimensiones utilizando los algoritmos de Oja y Sanger. En el segundo se planteó un modelo de auto\-organización a partir del algoritmo de Kohonen. Distintas configuraciones de parámetros fueron estudiadas en cada caso. Se presentan aquıí los resultados obtenidos.

\section{Ejercicio 1: Reducción de dimensiones}
\documentclass[onecolumn,10pt]{article}
\linespread{1.3}
\usepackage[margin=2cm]{geometry}

\usepackage[utf8]{inputenc}
\usepackage[spanish]{babel}
\usepackage{tikz}
\usepackage{paralist}
\usepackage{subcaption} \usepackage{graphicx}
\usepackage{amsmath} \usepackage{amssymb}
\usepackage{xcolor}
\usepackage{listings}

\usepackage{amsmath}
\usepackage{algorithm}
\usepackage[noend]{algpseudocode}

\lstset{language=C}
\bibliographystyle{apsrev4-1}

\graphicspath{{./fig/}}

\newcommand{\IDT}{\texttt{IDT}}
\newcommand{\MMU}{\texttt{MMU}}
\newcommand{\GDT}{\texttt{GDT}}
\newcommand{\TSS}{\texttt{TSS}}
\newcommand{\PD}{\texttt{PD}}

\begin{document} 

\title{Clasificación de Resultados de Diágnostico de Cancer de Mama utilizando Perceptron Multicapa}

\author{L. Alvarex, P. Bordon y D. Santos}


\date{\today}

%\begin{abstract}
%\end{abstract}

\maketitle

\section{Introducción}
La primer etapa del trabajo practico consiste en la implementación
de una red neuronal para la clasificación de diagnostico de cancer
de mama a partir de un dataset con los resultados de analisis y
su diagnostico clasificados en \textbf{B} (benigno) o \textbf{M} (maligno).

El objetivo es determinar si es posible aplicar \emph{Redes Neuronales} para
diagnosticar de forma efectiva si un tumor es \textbf{Benigno} o \textbf{Maligno}.

Para lograr el objetivo presentaremos una \emph{Red Neuronal} basada en la
implementación de un \textbf{Perceptron Multicapa}.

Para dicha implementación las etapas desarrolladas son:

\begin{itemize}
\item Análisis de la Red: Describiremos los primeros enfoques a la
  solución propuesta
\item Prepocesamiento de datos: Proceso aplicado al dataset recibido.
\item Arquitectura definitiva: Implementación final de la arquitectura
  de la red.
\item Implementación de la Red: La implementación del perceptron y sus
   algoritmos.
\end{itemize}


\section{Análisis de la Red}

La primera aproximación a la solucion fue implementar una red de una
capa con 8 neuronas, la cantidad de neuronas elegida fue alta, 
entendiendo que podía generarnos el problema de \emph{overfitting}.
Siendo nuestro objetivo principal estudiar el aprendizaje de la red
para luego ir ajustando la cantidad de capas y neuronas.
Como función de activación elegimos la tangente hiperbólica. 
Contrario a lo esperado, nuestra primera red no lograba \emph{aprender}.
Obteníamos un \emph{Error Cuadrático Medio} alto en promedio. Y a pesar
de incrementar la cantidad de iteraciones se estancaba en valores
cercanos al 0.50 aproximadamente, haciendo imposible una correcta
clasificación.
Lo siguiente fue empezar a modificar la arquitectura de la red,
variando la cantidad de neuronas y al seguir sin mejoras, agregamos
sin éxito una segunda capa de neuronas.

Descubrimos que el motivo por el cual la red se comportaba de esta
forma se debía a que al aplicar la función de activación a los valores
del dataset estos alcanzaban el valor 1 siempre. Entonces para corregir
el problema realizamos un pre-procesamiento de los datos.


\section{Procesamiento de Datos}

Para solucionar el problema que genera aplicarle la función de activación
al dataset realizamos un pre procesamiento de los datos.
Este procesamiento consiste en \emph{normalizar} los datos con varianza = 0 y
sigma = 1.

Con este procesamiento nos aseguramos que aplicandole la función de
activación de tangente hiperbólica a los datos obtenemos valores entre
0 y 1.

\section{Arquitectura definitiva}

Con los datos procesados comenzamos con las pruebas para encontrar
una arquitectura lo suficientemente robusta para alcanzar un buen nivel
de aprendizaje pero que no caiga en la redundancia del \emph{overfitting}.

Implementamos una arquitectura de una sola capa interna, con 10 + 1 neuronas 
en la capa de entrada, 10 para los datos del data set y una inicializada en -1
y 1 neurona en la de salida. Para la capa interna implementamos
soluciones con distinta cantidad de neuronas en la segunda capa, tomando 
valores de 8 a 5 y realizamos pruebas variando los parametros de
 \emph{leanrning rate}, \emph{cantidad de iteraciones} y \emph{tolerancia de error}.

 Mas adelante presentaremos los resultados obtenidos en la sección de \textbf{Resultados}

 Luego de realizar los experimentos detectamos que la arquitectura que presenta
 mejor adaptación para el aprendisaje es la que implementa una capa interna de entre
 5 o 6 neuronas.

 Entonces la arquitectura definitiva propuesta es:

10 + 1 neuronas como \emph{entrada}, una capa \emph{interna} de 5 neuronas 
y 1 neurona para la capa de \emph{salida}.

Para entrenar la red lo hacemos con el 80 por ciento de los datos del dataset,
el resto los dejamos para validación.

Definimos una función de activación basada en la tangente hiperbólica para ambas
capas, pero dejamos la libertad de utilizar distintos parametros de beta en cada
una de ellas.

Una vez entrenada la red corremos un algoritmo de testoe basado en la técnica de
\textbf{cross-validation}.

El parametro \textbf{cantidad-mezclas} es el que define la cantidad de veces que
el algoritmo toma \emph{k-folds} del dataset para entrenar y validar.


\section{Implementación de la Red}

A continuación detallaremos como implementamos la red.

Implementamos una clase \textbf{Perceptron} con la siguiente estructura

\begin{lstlisting}
  struct perceptron {
    learning_rate 
    beta1 
    beta2
    tolerancia_error
    cantidad_repeticiones
    cantidad_mezclas
    input_file 
    output_file
    tamano_capa
    tamano_entrada
    tamano_salida 
    w1  
    w2  
  }

\end{lstlisting}


\begin{itemize}
\item \textbf{learning rate} = coeficiente de aprendizaje.
\item \textbf{beta1} = parametro beta en la primera función de activación.
\item \textbf{beta2} = parametro beta en la segunda función de activación.
\item \textbf{tolerancia-error} = tolerancia de error.
\item \textbf{cantidad repeticiones} = cantidad de epocas.
\item \textbf{cantidad mezclas} = cantidad de veces que se ejecutará.
\item \textbf{input file} = archivo de entrada.
\item \textbf{output file} = archivo de salida.
\item \textbf{tamano capa} = tamaño de capa interna.
\item \textbf{tamano entrada} = tamaño de capa de entrada.
\item \textbf{tamano salida} = tamaño de capa de salida.
\item \textbf{w1} = vector de pesos de la primer capa.
\item \textbf{w2} = vector de pesos de la segunda capa.
\end{itemize}


Definida la estructura principal del \textbf{Perceptron} presentamos
las funciones principales que utilizá la \emph{Red Neuronal} para
la clasificación de datos.


\begin{itemize}
\item \textbf{entrenar} = entreniento de la red, realiza el preprocesamiento
del dataset, y para la cantidad seteada de mezclas realiza un entrenamiento, tomando
como cota la cantidad de iteraciones y la tolerancia del error. Dentro del ciclo principal
calcula la activación, corrección y adaptación de la red. Luego realiza cross-validation para
verificar los resultados de cada época.
\item \textbf{testing} = toma una red entrenada y calcula la tasa de aciertos.
\item \textbf{funcion activacion} = funcion de activación.
\item \textbf{funcion activacion derivada} = funcion de activación derivada.
\end{itemize}


\section{Experimentación y Resultados}



Definida la red procedemos a realizar  un entrenamiento de la misma 
variando el tamaño de la capa interna, entrenamos el suficiente tiempo para
encontrar el punto donde empieza a converger el ECM.

La primera configuración es la siguiente

\begin{itemize}
\item \textbf{learning-rate} = 0.1
\item \textbf{beta1} = 0.1
\item \textbf{beta2} = 0.1
\item \textbf{tolerancia-error} = 1.
\item \textbf{cantidad-repeticiones} = 10000.
\item \textbf{cantidad-mezclas} = 10.
\item \textbf{tamano-capa} = 5.
\end{itemize}


Tomamos la salida con menor ECM y presentamos en un gráfico la convergencia
de los errores:

\begin{figure}[H]
  \centering
  \includegraphics[width=0.8\columnwidth]{red_5_ecm.png}
  \caption{Comparación contra ECM.}
  \label{fig:red 5 ECM}
\end{figure}



\begin{figure}[H]
  \centering
  \includegraphics[width=0.8\columnwidth]{red_5_prom.png}
  \caption{Comparación mínimo, promedio y máximo.}
  \label{fig:red promedios}
\end{figure}

Podemos observar que el ECM durante las primeras 2000 iteraciones fue descendiendo
y a partir de la iteracion 2000 el descenso fue menor, pero siempre en baja.
No alcanzamos un punto donde empieze a oscilar.

La comparativa de los errores Promedio, Mínimo y Máximo muestra que se mantuvieron
casi constante y no variaron a pesar de las 10000 iteraciones.

Procedemos a ejecutar la función test para ver que resultados produjo el entrenamiento 

La tasa de aciertos luego de ejecutar la función testing se úbico 367 en sobre 410.

Adjuntamos la red entrenada en el archivo: red-5-train.in.



Repetimos el mismo experimento para una red de 6 neuronas:

\begin{itemize}
\item \textbf{learning-rate} = 0.1
\item \textbf{beta1} = 0.1
\item \textbf{beta2} = 0.1
\item \textbf{tolerancia-error} = 1.
\item \textbf{cantidad-repeticiones} = 10000.
\item \textbf{cantidad-mezclas} = 10.
\item \textbf{tamano-capa} = 6.
\end{itemize}


Presentamos los resultados correspondientes a la mezcla que produjo
el menor ECM:

\begin{figure}[H]
  \centering
  \includegraphics[width=0.8\columnwidth]{red_6_ecm.png}
  \caption{Comparación contra ECM.}
  \label{fig:red 6 ECM}
\end{figure}


\begin{figure}[H]
  \centering
  \includegraphics[width=0.8\columnwidth]{red_6_prom.png}
  \caption{Comparación mínimo, promedio y máximo.}
  \label{fig:red promedios}
\end{figure}

La comparativa nos muetsra que el error Máximo oscila mucho en las primeras 
iteraciones y luego se estabiliza y se mantiene constante.

El ECM promedio fue de 7.008, es decir relativamente superior al menor de la red
de 5 neuronas en la capa interna. De nuevo vemos que a partir de la iteración 2000
comienza a descender muy lentamente.

La tasa de aciertos luego de ejecutar la función testing se úbico en 370 sobre 410.
Siendo apenas superior por a la red de 5 neuronas, esto se reduce a esta instancia 
particular no hay garantía de que siempre sea mejor.


Adjuntamos la red entrenada en el archivo: red-6-train.in.

Ahora realizamos el mismo experimento para una red de 4 neuronas:

\begin{itemize}
\item \textbf{learning-rate} = 0.1
\item \textbf{beta1} = 0.1
\item \textbf{beta2} = 0.1
\item \textbf{tolerancia-error} = 1.
\item \textbf{cantidad-repeticiones} = 10000.
\item \textbf{cantidad-mezclas} = 10.
\item \textbf{tamano-capa} = 4.
\end{itemize}


Presentamos los resultados correspondientes a la mezcla que produjo
el menor ECM:

\begin{figure}[H]
  \centering
  \includegraphics[width=0.7\columnwidth]{red_4_ecm.png}
  \caption{Comparación contra ECM.}
  \label{fig:red 4 ECM}
\end{figure}


\begin{figure}[H]
  \centering
  \includegraphics[width=0.7\columnwidth]{red_4_prom.png}
  \caption{Comparación mínimo, promedio y máximo.}
  \label{fig:red promedios}
\end{figure}

El ECM promedio que se obtuvo fue de 8,67 muy superior a las otras redes, además se
puede notar que el ECM oscila mucho entre las iteraciones, es decir no siempre desciende.

En la comparativa de los errores se puede ver que al máximo le lleva muchas iteraciones
estabilizarse, aunque no lo logra del todo. El promedio y el mínimo tambiém se mantienen
oscilantes

La tasa de aciertos luego de ejecutar la función testing se úbico en 355 sobre 410.

Damos por descartado que la red con una capa interna de 4 neuronas sea la mejor opción.
De todas formas adjuntamos la red en el archivo red-4-train.in

De los experimentos realizados la red con 5 neuronas fue la que a priori puede brindar
mejores resultados es la que tiene 5 neuronas en la capa interna. Nos resulto ser una red lo suficientemente robusta para aprender sin caer en el riesgo de \emph{overfitting}.

Vamos a estudiar que tan bien responde con instancias mas cortas de entrenamiento. Ahora vamos
entrenar la red con los siguientes parámetros:

\begin{itemize}
\item \textbf{learning-rate} = 0.1
\item \textbf{beta1} = 0.1
\item \textbf{beta2} = 0.1
\item \textbf{tolerancia-error} = 1.
\item \textbf{cantidad-repeticiones} = 500.
\item \textbf{cantidad-mezclas} = 5.
\item \textbf{tamano-capa} = 5.
\end{itemize}

El objetivo es ver si con menos instancia de entrenamiento adquiere una proporción
similar de resultados.

Presentamos los resultados correspondientes a la mezcla que produjo
el menor ECM:


\begin{figure}[H]
  \centering
  \includegraphics[width=0.7\columnwidth]{red_5_ecm_reducida.png}
  \caption{Comparación contra ECM.}
  \label{fig:red 5 ECM}
\end{figure}


\begin{figure}[H]
  \centering
  \includegraphics[width=0.7\columnwidth]{red_5_prom_reducida.png}
  \caption{Comparación mínimo, promedio y máximo.}
  \label{fig:red promedios}
\end{figure}

El ECM promedio se ubico en 6.97

Vemos que los graficos de errores se acomodan rapidamente tendiendo a valores obtenidos
con mas iteraciones-

La tasa de aciertos luego de ejecutar la función testing se úbico en 387 sobre 410.

Repetimos el experimento varias veces y en todos los casos obtuvimos aciertos entre los
382-397, nos hace pensar que entrenar con iteraciones de mas no siempre es lo mejor.
Ya que podriamos estas sobreentrenando la red, causando que esta memorize los valores
y pierda el sentido de la predicción.

Adjuntamos la red en el archivo red-5-train-reducida.in


\section{Concluciones}

La \textbf{Red Neuronal} propuesta basada en una implementación del Perceptrón Multicapa,
logro en gran parte de las ejecuciones un alto porcentaje de aciertos, sobre todo a 
partir del último experimento.
Esta técnica puede ser util para ayudar en el diagnostico del cancer de mama, pero
no debe ser la única aplicada. Ya que la información y el veredicto inciden
sobre la vida de la persona diagnosticada, seria prudente cotejar los resultados
con técnicas mas sofisticadas de dianóstico.


\newpage


\end{document}


\section{Ejercicio 2: Mapeo de características}

\section{Mapeo de Características}

Esta etapa del trabajo consiste en la implementación de un \textbf{mapa
de características} para la clasificación de las palabras según
la \textbf{categoría} que les fue asignada previamente.

Lo que se espera obtener es un \textbf{mapa auto-organizado} en el cual
cada \textbf{categoría} este claramente diferenciada.

Para lograr el objetivo presentaremos un \textbf{mapa auto-organizado} 
basado en el algoritmo de \emph{Kohonen}.

Detalleremos los pasos que realizamos para la implementación de la solución,
la elección de la implementación y las dificultades que nos topamos al
implementarlo.

Dividimos el informe en las siguientes secciones:

\begin{itemize}
\item Implementación del Algoritmo: detallamos de forma introductoria la
implementación del algoritmo. En las siguientes etapas discutiremos los pasos
hacia la implementación final.
\item Dimensiones Adecuadas: discutimos las alternativas para la dimensión
del mapa.
\item Vecindad, Learning Rate y Sigma: discutimos las alternativas que
estudiamos para seleccionar los parámetros.
\item 
\end{itemize}

\subsection{Implementación del Algoritmo}

Para implementar la solución tomamos el algoritmo de mapas
de \emph{Kohonen} visto en clase.

La versión que implementamos es la que realiza los calculos 
por \textbf{columna}, esta versión demora mas en los calculos 
que la versión matricial, pero su implementación nos resulto mas 
sencilla.

Tomamos una dimensíón del mapa de 20 x 20 y para la actualización 
de las vecindades utilizamos la \textbf{función gaussiana} con un 
\textbf{learning rate adaptativo}. Comenzando con un radio de 
vecindad de 10.

Para el entrenamiento utilizamos dos tipos de cota, una por \textbf{cantidad
de epocas} y la otra por \textbf{la diferencia de la norma de la matriz de pesos
entre dos epocas distintas}, cuando es menor a un delta se finaliza 
el entrenamiento.


\subsection{Dimensiones Adecuadas}

Buscamos una dimensión adecuada para el mapa, conociendo que queremos
clasificar 900 datos en 9 categorías.

Probamos con las siguientes dimensiones:

\begin{itemize}
	\item 10 x 10
	\item 20 x 20
	\item 30 x 30
\end{itemize}

\subsubsection{Dimensión 10 x 10 }

Obtuvimos que el mapa no era lo suficientemente grande y los valores de
entrada se solapaban mucho, como lo muestra la figura:

\begin{figure}[H]
  \centering
  \includegraphics[width=0.8\columnwidth]{red_5_ecm.png}
  \caption{Mapa 10 x 10 para 250 entradas.}
  \label{fig:mapa 10 10 250}
\end{figure}


Por lo tanto descartamos esta configuración

\subsubsection{Dimensión 20 x 20 }

Con estas dimensiones obtuvimos una buena distribución de las categorías,
aunque hay un porcentaje de neuronas que no se activaron.

\begin{figure}[H]
  \centering
  \includegraphics[width=0.8\columnwidth]{red_5_ecm.png}
  \caption{Mapa 10 x 10 para 250 entradas.}
  \label{fig:mapa 10 10 250}
\end{figure}


\subsubsection{Dimensión 30 x 30 }

Para esta configuración obtuvimos mayor desperdicio de neuronas,
es decir el porcentaje que nunca se activo fue muy alto, como lo muestra
la siguiente figura

\begin{figure}[H]
  \centering
  \includegraphics[width=0.8\columnwidth]{red_5_ecm.png}
  \caption{Mapa 10 x 10 para 250 entradas.}
  \label{fig:mapa 10 10 250}
\end{figure}


\subsubsection{Conclusión sobre la Dimensiones}


La que mejor se adapta a las características del problema, es la
de dimensión de 20 x 20.


\subsection{Vecindad, Learning Rate y Sigma}

Para la elección de estos parametros consideramos las siguientes
alternativas:

\textbf{Vecindad} por:

\begin{itemize}
	\item Escalones: influenciamos vecinos a distancia n formando
un cuadrado.
	\item Función Gaussiana: influenciamos n vecinos utilizando
una función gaussiana.
\end{itemize}


\textbf{Learning Rate} por:

\begin{itemize}
	\item Basado en la cantidad de epocas: definimos el learning
rate basado en la cantidad de epocas aplicadas a una función.
	\item Coeficientes adaptativos: definimos un valor de
learning rate inicial y lo disminuimos a traves de coeficientes.
	\item Decrecimiento porcentual: Definimos un valor de 
learning rate inicial y por epoca lo decrementamos en un 0.X por ciento.
\end{itemize}

Para el \emph{Learning Rate} basados en la \textbf{Cantidad de Epocas}, probamos con 
las siguientes funciones:

\begin{itemize}
	\item \frac{1}{t^\frac{-1}{2}}
	\item \frac{1}{t^\frac{-1}{3}}
\end{itemize}

En el caso de \textbf{Coeficientes Adaptativos}, probamos con:
learning rate inicial / (1 + epoca_actual * coeficiente * learning rate inicial)

Y para del \textbf{Decrecimiento Porcentual}: por cada epoca learning rate inicial * 0.X


\textbf{Sigma} por:

Definimos un valor de sigma_inicial, lo suficientemente grande para influenciar a todo 
el mapa. Y variandolo con las epocas con dos alternativas:

\begin{itemize}
	\item sigmaInicial * {epocaActual^\frac{-1}{3}}
	\item sigmaInicial * e^\frac{epocaActual}{lambd}
\end{itemize}

Tomando como lambd: cantidadEpocas * cantidadDatosEntrenamiento / log (sigmaInicial)


\subsubsection{Buscando la Combinación Ideal}

Realizamos una bateria de prueba tomando un set de datos de 300 valores, para
estudiar cual de la siguiente combinación de parametros organizaba mejor
las categorías.

Primero definimos un sigma inicial: (dimensión mapa) / 2, así en las primeras
epocas gran parte del mapa seria influenciado.

De las primeras pruebas concluimos que el \emph{Learning Rate} basado en la 
\textbf{cantidad de epocas} resultaba no ser una buena elección, ya que
decrece de forma muy rápido sin dar el tiempo suficiente para aprender.
Entonces descartamos esta opción.

Entonces utilizamos las otras dos opciones, definiendo un \emph{Learning Rate}
alto con valor de 0.999.En el caso de \textbf{Decrecimiento Porcentual}
tuvimos el inconveniente de que o decrecía muy rápido o demasiado lento, haciendo
dificil encontrar el punto justo.

El que nos permitió encontrar el balance fue \textbf{Coeficientes Adaptativos}
tomando como coeficiente = 0.5.

La función de \textbf{Sigma} basada en sigmaInicial * e^\frac{epocaActual}{lambd}
nos dió que decrecia de forma muy rápido, impidiendo el aprendisaje.
La función sigmaInicial * {epocaActual^\frac{-1}{3}} combinado al \emph{Learning Rate}
de tipo \textbf{adaptativo} nos dió un buen balance.

Para la función de vecindad decidimos quedarnos con la \textbf{función uniforme} ya
que presentaba una distribución más ármonica.

\subsubsection{Conclusiones de los Parámetros}

La combinación que nos resulto mejor fue tomar:


\begin{itemize}
	\item Sigma Inicial = 10
	\item Learning Rate Adaptativo = 0.999 / (1 + epoca_actual * 0.5 * 0.99)
	\item Sigma = Sigma Inicial * {epocaActual^\frac{-1}{3}}
\end{itemize}


\subsection{Cotas}

Para finalizar el entrenamiento definimos dos tipos de cotas:

\begin{itemize}
	\item Epocas
	\item Norma
\end{itemize}

La basada en la cantidad de \textbf{Epocas} establece una cantidad máxima de
ciclos de entrenamiento.

La basada en la \textbf{Norma} calcula la norma de la matriz de pesos (W) 
correspondiete a la epoca actual, hace la diferencia con la epoca anterior.
En caso que sea menor a una epsilon termina el entrenamiento.


\subsection{Entrenamiento}


\subsection{Detalles de Implementación}

\subsection{Detalles de Resultados}

Los resultados correspondientes a la sección de entrenamiento son adjuntados con 
la siguiente estructura:

XXX Entradas/Etapa X/: la carpeta Entradas corresponde a la cantidad de datos usados para entrenar la red
y la subcarpeta Etapa corresponde a una etapa de entrenamiento. Dentro de la carpeta de
Etapas se encuentran los siguientes archivos

\begin{itemize}
	\item mapa XXX X.in =  corresponde al mapa entrenado.
	\item resultados mapa XXX X.txt = corresponde a los resultados de ejecutar la
validación sobre el entrenamiento.
	\item mapa XXX X aciertos.png = muestra la posición en la mapa de los registros bien clasicados.
	\item mapa XXX X errores.png = muestra la posición en la mapa de los registros mal clasicados.
	\item mapa XXX X indefinidas.png = = muestra la posición en la mapa de los registros que no
se han podido clasificar.
	\item mapa XXX X entrenamiento = es el mapa obtenido del entrenamiento.
\end{itemize}

\subsection{Conclusiones}



\end{document}