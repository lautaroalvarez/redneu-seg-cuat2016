\section{Descripción del problema}

\par Se cuenta con documentos de descripción de empresas. Dichos documentos se encuentran preprocesados, de manera tal que se tiene un dataset que especifica qué cantidad de veces aparece cada palabra en cada documento. Este formato es conocido como Bolsa de Palabras (Bag-Of-Words). El problema a resolver será diseñar y entrenar una red neuronal mediante métodos de aprendizaje hebbiano no supervisado para que pueda clasificar automáticamente los documentos según categoría. También se busca que esta red pueda generalizar corréctamente y responda de manera certera a nuevos documentos.


\par Se utilizaron dos subparadigmas clásicos de aprendizaje no supervisado: aprendizaje hebbiano y aprendizaje competitivo. Se trabajó en un modelo de reducción de dimensiones y otro de mapeo de características auto\-organizado para su aplicación sobre las datos propuestos. En el primer caso se planteó una reducción a tres dimensiones utilizando los algoritmos de Oja y Sanger. En el segundo se planteó un modelo de auto\-organización a partir del algoritmo de Kohonen. Distintas configuraciones de parámetros fueron estudiadas en cada caso. Se presentan aquıí los resultados obtenidos.