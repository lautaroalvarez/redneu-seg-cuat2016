\documentclass[10pt,a4paper]{article}
\usepackage[utf8]{inputenc} % para poder usar tildes en archivos UTF-8
\usepackage[spanish]{babel} % para que comandos como \today den el resultado en castellano
%\usepackage{a4wide} % márgenes un poco más anchos que lo usual
\usepackage{Sty/caratula}
%\usepackage{Sty/mathtools}
%\usepackage{Sty/float}
%\usepackage[pdftex]{graphicx}
%\usepackage{caption}
%\usepackage{subcaption}
%\usepackage[ruled,vlined,linesnumbered]{Sty/algorithm2e}
%Esto de abajo es para encabezado y pie de pagina
%\usepackage{Sty/lastpage}
%\usepackage{fancyhdr}
%\usepackage{amsfonts}
%\usepackage[noend]{algpseudocode}
%\usepackage{enumerate} % AGREGO PARA PODER ENUMERAR LAS LINEAS DEL ALGORITMO
%\usepackage{wrapfig}
%\usepackage{amsmath}
%\usepackage{verbatim}
%\usepackage{listings}
%\usepackage{color}
%\usepackage{dsfont}




%\cfoot{\thepage /\pageref{LastPage} }


\begin{document}

%\fecha{\today}
\materia{Redes Neuronales}
\titulo{Trabajo Práctico I}

\integrante{Diego Santos}{}{diego.h.santos@gmail.com}
\integrante{Pablo Bordon}{}{bordonpablo@gmail.com}
\integrante{Lautaro Leonel Alvarez}{268/14}{lautarolalvarez@gmail.com}

\maketitle
\newpage
\tableofcontents	

\newpage
%\section{Problema 1}
%\input{Problema1/problema1.tex}

\end{document}
